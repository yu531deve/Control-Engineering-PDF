% --------------- [41] ---------------
\begin{tikzpicture}[auto, node distance=0.8cm and 1.0cm, >=Latex]

    % --- ノードと信号配置(横並び) ---
    \node at (0,0) (R) {$R$};                           % R(目標値)
    \node[block, right=of R] (G3) {$G_3$};              % G3
    \node[circle, draw, inner sep=1.5pt, right=of G3] (sum1) {};  % 合流点1
    \node[block, right=of sum1] (G1) {$G_1$};           % G1
    \node[circle, draw, inner sep=1.5pt, right=of G1] (sum2) {};  % 合流点2
    \node[block, right=of sum2] (G2) {$G_2$};           % G2
    \node[circle, fill=black, inner sep=1.5pt, right=of G2] (branch) {}; % 分岐点
    \node at ($(branch)+(1.2,0)$) (Y) {$Y$};             % Y(制御量)
  
    % --- 上下ノード ---
    \node at ($(sum2)+(0,1.5)$) (D) {$D$};               % D(外乱)
    \node[block, below=1.0cm of sum2] (H1) {$H_1$};      % H1(下)
  
    % === 経路 ===
    \draw[->] (R) -- (G3);
    \draw[->] (G3) -- (sum1);
    \draw[->] (sum1) -- (G1);
    \draw[->] (G1) -- (sum2);
    \draw[->] (sum2) -- (G2);
    \draw[->] (G2) -- (branch);
    \draw[->] (branch) -- (Y);
  
    \draw[->] (D) -- (sum2);            % D → 合流点2
    \draw[->] (branch) |- (H1);         % 分岐 → H1(下)
    \draw[->] (H1) -| (sum1);           % H1 → 合流点1(戻る)
  
    % --- 加算記号配置 ---
    \node at ($(sum1)+(-0.3,0.3)$) {\small $+$};
    \node at ($(sum1)+(-0.3,-0.3)$) {\small $-$};
    \node at ($(sum2)+(-0.3,0.3)$) {\small $+$};
    \node at ($(sum2)+(-0.3,-0.3)$) {\small $+$};
  
\end{tikzpicture}