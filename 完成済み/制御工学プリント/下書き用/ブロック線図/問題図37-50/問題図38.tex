
\begin{tikzpicture}[auto, node distance=1.2cm and 0.8cm, >=Latex]

    % ノード定義
    \node[input] (input) {};
    \node[circle, fill=black, inner sep=1.5pt, minimum size=3pt, right=of input] (branch) {}; % 分岐点:黒丸(サイズ修正)

    \node[block, right=of branch] (G1) {$G_1$};
    \node[circle, draw, inner sep=1.5pt, minimum size=3pt, right=of G1] (sum1) {}; % 中間合流点:白丸

    \node[block, right=of sum1] (G2) {$G_2$};
    \node[circle, draw, inner sep=1.5pt, minimum size=3pt, right=of G2] (sum2) {}; % 出力直前合流点:白丸

    \node[output, right=of sum2] (output) {};

    \node[circle, fill=black, inner sep=1.5pt, minimum size=3pt, below=1.5cm of sum1] (branch2) {}; % 下の黒丸(サイズ修正)
    

    % 線描画
    \draw[-] (input) -- (branch);
    \draw[->] (branch) -- (G1);
    \draw[->] (G1) -- (sum1);
    \draw[->] (sum1) -- (G2);
    \draw[->] (G2) -- (sum2);
    \draw[->] (sum2) -- (output);

    \draw[-] (branch) |- (branch2);
    \draw[->] (branch2.north) -| (sum1);

    \draw[-] (branch2.east) -- ++(3.0,0); % .east を使うと右側から出る
    \draw[->] ($(branch2.east)+(3.0,0)$) - | (sum2);
    
    % 加算記号を合流点にそれぞれ配置
    \node at ($(sum1)+(-0.3,0.3)$) {\small $+$};
    \node at ($(sum1)+(-0.3,-0.3)$) {\small $+$};

    \node at ($(sum2)+(-0.3,0.3)$) {\small $+$};
    \node at ($(sum2)+(-0.3,-0.3)$) {\small $+$};

\end{tikzpicture}