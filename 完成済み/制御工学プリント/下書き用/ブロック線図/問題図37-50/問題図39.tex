% --------------- [39] ---------------
\begin{tikzpicture}[auto, node distance=1.0cm and 1.8cm, >=Latex]

    % --- 上段ノード ---
    \node[input] (input) {};
    \node[circle, draw, inner sep=1.5pt, right=of input] (sum1) {};              % 合流点①
    \node[block, right=of sum1] (G) {$G$};                                       % G
    \node[circle, fill=black, inner sep=1.5pt, right=of G] (branch1) {};         % 分岐点①
    \node[output, right=of branch1] (output) {};                                 % 出力

    % --- H₁・H₂ 縦並び(Gの真下) ---
    \node[block, below=1.0cm of G] (H1) {$H_1$};
    \node[block, below=0.5cm of H1] (H2) {$H_2$};
    \coordinate (midH) at ($(H1)!0.5!(H2)$);

    % --- 分岐点②(右):H₁とH₂の中間、高さmidH、X位置はGとbranch1の中間よりやや右
    \path (G) -- (branch1) coordinate[pos=0.6] (betweenGB);
    \node[circle, fill=black, inner sep=1.5pt] at ($(midH)+(betweenGB)-(G)+(0.3,0)$) (branch2) {}; % 分岐点②

    % --- 合流点②(左):midH高さ、X位置はHₛとGの中間
    \node[block, below=0.5cm of sum1] (Hs) {$H_s$};                              % H_s
    \path (sum1) -- (G) coordinate[pos=0.5] (leftX);
    \node[circle, draw, inner sep=1.5pt] at ($(midH)+(leftX)-(G)+(-0.3,0)$) (sum2) {};  % 合流点②

    % === 経路 ===
    \draw[->] (input) -- (sum1);
    \draw[->] (sum1) -- (G);
    \draw[->] (G) -- (branch1);
    \draw[->] (branch1) -- (output);

    \draw[->] (branch1) |- (branch2);                 % 分岐1 → 分岐2
    \draw[->] (branch2) |- (H1);                      % 分岐2 → H1(上方向)
    \draw[->] (branch2) |- (H2);                      % 分岐2 → H2(下方向)

    \draw[->] (H1) -| (sum2);                         % H1 → 合流2
    \draw[->] (H2) -| (sum2);                         % H2 → 合流2

    \draw[->] (sum2) -| (Hs);                         % 合流2 → Hs
    \draw[->] (Hs) -- (sum1);                         % Hs → 合流1(フィードバック)

    % --- 加算記号(揃え済) ---
    \node at ($(sum1)+(-0.3,0.3)$) {\small $+$};
    \node at ($(sum1)+(-0.3,-0.3)$) {\small $-$};
    \node at ($(sum2)+(0.3,0.3)$) {\small $+$};
    \node at ($(sum2)+(0.3,-0.3)$) {\small $-$};

\end{tikzpicture}