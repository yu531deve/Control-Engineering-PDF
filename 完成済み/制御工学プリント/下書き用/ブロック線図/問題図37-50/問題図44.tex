% --------------- [44] ---------------
\begin{tikzpicture}[auto, node distance=0.8cm and 1.0cm, >=Latex]

    % --- 上段ノード(横並び) ---
    \node[circle, draw, inner sep=1.5pt] (sum1) {};                % 合流点1
    \node[block, right=of sum1] (G1) {$G_2$};                      % G1
    \node[circle, fill=black, inner sep=1.5pt, right=of G1] (branch) {}; % 分岐点
  
    % input1とoutputの位置(描かない)
    \coordinate[left=of sum1] (input1);
    \coordinate[right=of branch] (output);
  
    % --- 中央の縦ノード(合流1 → G2 → 合流2) ---
    \node[block, below=0.8cm of sum1] (G2) {$G_1$};                % G2
    \node[circle, draw, inner sep=1.5pt, below=0.8cm of G2] (sum2) {}; % 合流点2
  
    % --- 下段(横並び) ---
    \coordinate[left=of sum2] (input2);                            % input2(描かない)
    \node[block, right=of sum2] (H) {$H$};                         % H
  
    % === 経路 ===
    \draw[->] (input1) -- (sum1);          % input1 → 合流点1
    \draw[->] (sum1) -- (G1);              % 合流点1 → G1
    \draw[->] (G1) -- (branch);            % G1 → 分岐点
    \draw[->] (branch) -- (output);        % 分岐点 → 出力
  
    \draw[->] (branch) |- (H);             % 分岐点 → H
    \draw[->] (H) -- (sum2);               % H → 合流点2
    \draw[->] (sum2) -- (G2);              % 合流点2 → G2
    \draw[->] (G2) -- (sum1);              % G2 → 合流点1
  
    \draw[->] (input2) -- (sum2);          % input2 → 合流点2
  
    % --- 加算記号配置 ---
    \node at ($(sum1)+(-0.3,0.3)$) {\small $+$};
    \node at ($(sum1)+(-0.3,-0.3)$) {\small $+$};
    \node at ($(sum2)+(-0.3,-0.3)$) {\small $+$};
    \node at ($(sum2)+(0.3,-0.3)$) {\small $-$};

    % --- その他記号配置 ---
    \node at ($(sum1)+(-1.0,0.3)$) {\small $a_2(s)$};
    \node at ($(sum2)+(-1.0,0.3)$) {\small $a_1(s)$};
    \node at ($(output)+(-0.3,0.3)$) {\small $b(s)$};
  
\end{tikzpicture}