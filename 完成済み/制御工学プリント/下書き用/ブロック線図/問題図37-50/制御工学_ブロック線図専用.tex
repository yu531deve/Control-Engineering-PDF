\documentclass[a4paper,12pt]{article}
%\documentclass[fleqn]{article}

% ---パッケージ---
\usepackage{amsmath,amssymb}    %数式用
\usepackage{tcolorbox}   %囲み枠用(tcolorboxに変更)
\usepackage{geometry}   %余白調節
\usepackage{tikz}  % ← 図を描くためのTikZパッケージ
\geometry{margin=25mm}  %余白を少し狭く
\usetikzlibrary{decorations.pathmorphing,patterns,positioning,arrows.meta} % バネ・壁の模様
\tikzset{
  block/.style = {draw, rectangle, minimum height=2em, minimum width=3em},
  sum/.style = {draw, circle, inner sep=0pt, minimum size=5mm},
  input/.style = {coordinate},
  output/.style = {coordinate}
}
\usetikzlibrary{calc}




% --- 日本語用パッケージ ---
\usepackage{luatexja}         % 日本語表示に必要
\usepackage{luatexja-fontspec} % フォント指定用

% --- フォント指定(Overleaf標準フォント)---
\setmainjfont{IPAexMincho}  % 明朝体
%\setmainjfont{IPAexGothic}  % ゴシック体にしたい場合

% --- tcolorbox の設定 ---
\tcbset{
    colframe=black,
    colback=white,         % 本文の背景(白)
    boxrule=0.8pt,
    arc=3pt,
    outer arc=3pt,
    boxsep=4pt,
    coltitle=black,
    colbacktitle=gray!20,  % タイトルの背景(グレー)
    fonttitle=\normalsize
}

\begin{document}


% --------------- [37] ---------------
37
\begin{center}
    \input{"問題図37.tex"}
\end{center}

% --------------- [38] ---------------
38
\begin{center}
    \input{"問題図38.tex"}
\end{center}

% --------------- [39] ---------------
39
\begin{center}
    \input{"問題図39.tex"}
\end{center}

% --------------- [40] ---------------
40
\begin{center}
    \input{"問題図40.tex"}
\end{center}
% --------------- [41] ---------------
41
\begin{center}
    \input{"問題図41.tex"}
\end{center}

\newpage

% --------------- [42] ---------------
42
\begin{center}
    \input{"問題図42.tex"}
\end{center}

% --------------- [43] ---------------
43
\begin{center}
    \input{"問題図43.tex"}
\end{center}
% --------------- [44] ---------------
44
\begin{center}
    \input{"問題図44.tex"}
\end{center}
% --------------- [45] ---------------
45
\begin{center}
    \input{"問題図45.tex"}
\end{center}
% --------------- [46] ---------------
46
\begin{center}
    \input{"問題図46.tex"}
\end{center}
% --------------- [47] ---------------
47
\begin{center}
    \input{"問題図47.tex"}
\end{center}
% --------------- [48] ---------------
48
\begin{center}
    \input{"問題図48.tex"}
\end{center}
% --------------- [49] ---------------
49
\begin{center}
    \input{"問題図49.tex"}
\end{center}
% --------------- [50] ---------------
50
\begin{center}
    \input{"問題図50.tex"}
\end{center}
\end{document}