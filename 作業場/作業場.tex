\documentclass[a4paper,12pt]{article}
%\documentclass[fleqn]{article}

% ---パッケージ---
\usepackage{amsmath,amssymb}    %数式用
\usepackage{tcolorbox}   %囲み枠用(tcolorboxに変更)
\usepackage{geometry}   %余白調節
\usepackage{tikz}  % ← 図を描くためのTikZパッケージ
\geometry{margin=25mm}  %余白を少し狭く
\usetikzlibrary{decorations.pathmorphing,patterns,positioning,arrows.meta} % バネ・壁の模様
\tikzset{
  block/.style = {draw, rectangle, minimum height=2em, minimum width=3em},
  sum/.style = {draw, circle, inner sep=0pt, minimum size=5mm},
  input/.style = {coordinate},
  output/.style = {coordinate}
}
\usetikzlibrary{calc}


% --- 日本語用パッケージ ---
\usepackage{luatexja}         % 日本語表示に必要
\usepackage{luatexja-fontspec} % フォント指定用

% --- フォント指定(Overleaf標準フォント)---
\setmainjfont{IPAexMincho}  % 明朝体
%\setmainjfont{IPAexGothic}  % ゴシック体にしたい場合

% --- tcolorbox の設定 ---
\tcbset{
    colframe=black,
    colback=white,         % 本文の背景(白)
    boxrule=0.8pt,
    arc=3pt,
    outer arc=3pt,
    boxsep=4pt,
    coltitle=black,
    colbacktitle=gray!20,  % タイトルの背景(グレー)
    fonttitle=\normalsize
}

\begin{document}

\begin{minipage}{0.5\linewidth}
\begin{center}
\begin{tikzpicture}[scale=0.3,xscale=3.5, yscale=0.05]

  % 軸
  \draw[->] (-0.2, -80) -- (4.5, -80) node[right] {$\log_{10} \omega$};
  \draw[->] (0, -85) -- (0, 70) node[above] {ゲイン [dB]};

  % 縦軸目盛
  \foreach \y in {-80, -60, -40, -20, 0, 20, 40, 60} {
    \draw (-0.05, \y) -- (0.05, \y);
    \node[left] at (-0.1, \y) {\y};
  }

% 横軸目盛(0.2だけ上に表示)
\foreach \x/\label in {
  0/0.01,
  1/0.1,
  1.3/0.2,
  2/1,
  2.7/5,
  3/10,
  4/100
} {
  \draw (\x, -82) -- (\x, -78);

  % 0.2 (x = 1.3) だけ上に表示
  \ifdim\x pt=1.3pt
    \node[above] at (\x, -82) {\scriptsize \label};
  \else
    \node[below] at (\x, -82) {\scriptsize \label};
  \fi
}


  % 折れ点(log10 ω = 0 ⇒ x = 2)
  \draw[dashed] (2, -80) -- (2, 60);
  \node[below] at (2, -82) {\scriptsize 1};

  % --- ゲイン漸近線(正確) ---
  \draw[thick] (0, 60) -- (2, 20);    % ω = 0.01 ~ 1(−20 dB/dec)
  \draw[thick] (2, 20) -- (4, -60);   % ω = 1 ~ 100(−40 dB/dec)

  \draw[thick, red]
  plot coordinates {
    (0.000, 60.0)
    (0.082, 58.4)
    (0.163, 56.7)
    (0.245, 55.1)
    (0.327, 53.5)
    (0.408, 51.8)
    (0.490, 50.2)
    (0.571, 48.6)
    (0.653, 46.9)
    (0.735, 45.3)
    (0.816, 43.7)
    (0.898, 42.0)
    (0.980, 40.4)
    (1.061, 38.7)
    (1.143, 37.1)
    (1.224, 35.4)
    (1.306, 33.7)
    (1.388, 32.0)
    (1.469, 30.3)
    (1.551, 28.5)
    (1.633, 26.6)
    (1.714, 24.7)
    (1.796, 22.6)
    (1.878, 20.5)
    (1.959, 18.2)
    (2.041, 15.7)
    (2.122, 13.1)
    (2.204, 10.4)
    (2.286, 7.5)
    (2.367, 4.6)
    (2.449, 1.5)
    (2.531, -1.6)
    (2.612, -4.7)
    (2.694, -7.9)
    (2.776, -11.1)
    (2.857, -14.4)
    (2.939, -17.6)
    (3.020, -20.9)
    (3.102, -24.1)
    (3.184, -27.4)
    (3.265, -30.6)
    (3.347, -33.9)
    (3.429, -37.1)
    (3.510, -40.4)
    (3.592, -43.7)
    (3.673, -46.9)
    (3.755, -50.2)
    (3.837, -53.5)
    (3.918, -56.7)
    (4.000, -60.0)
  };


\end{tikzpicture}

\end{center}



\begin{center}
\begin{tikzpicture}[scale=0.3,xscale=3.5, yscale=0.025]

  % 軸
  \draw[->] (-0.2, -210) -- (4.5, -210) node[right] {$\log_{10} \omega$};
  \draw[->] (0, -215) -- (0, 65) node[above] {位相 [deg]};

  % 縦軸目盛(60 → -210)
  \foreach \y in {-180, -90, 0, 60} {
    \draw (-0.05, \y) -- (0.05, \y);
    \node[left] at (-0.1, \y) {\y};
  }

  % 横軸目盛(0.2だけ上)
  \foreach \x/\label in {
    0/0.01,
    1/0.1,
    1.3/0.2,
    2/1,
    2.7/5,
    3/10,
    4/100
  } {
    \draw (\x, -212) -- (\x, -208);
    \ifdim\x pt=1.3pt
      \node[above] at (\x, -212) {\scriptsize \label};
    \else
      \node[below] at (\x, -212) {\scriptsize \label};
    \fi
  }

  % 折れ点(ω = 1 ⇒ x = 2)
  \draw[dashed] (2, -210) -- (2, 60);
  \node[below] at (2, -212) {\scriptsize 1};

  % --- 正しい位相漸近線 ---
  \draw[thick] (0, -90) -- (1.3, -90);         % ω = 0.01 ~ 0.2:位相 0°
  \draw[thick] (1.3, -90) -- (2.7, -180);    % ω = 0.2 ~ 5:下降(-180°へ)
  \draw[thick] (2.7, -180) -- (4, -180);   % ω = 5 ~ 100:定常 −180°


  \draw[thick, red]
  plot coordinates {
    (0.000, -90.6)
    (0.082, -90.7)
    (0.163, -90.8)
    (0.245, -91.0)
    (0.327, -91.2)
    (0.408, -91.5)
    (0.490, -91.8)
    (0.571, -92.1)
    (0.653, -92.6)
    (0.735, -93.1)
    (0.816, -93.7)
    (0.898, -94.5)
    (0.980, -95.5)
    (1.061, -96.6)
    (1.143, -97.9)
    (1.224, -99.5)
    (1.306, -101.4)
    (1.388, -103.7)
    (1.469, -106.4)
    (1.551, -109.6)
    (1.633, -113.2)
    (1.714, -117.4)
    (1.796, -122.0)
    (1.878, -127.0)
    (1.959, -132.3)
    (2.041, -137.7)
    (2.122, -143.0)
    (2.204, -148.0)
    (2.286, -152.6)
    (2.367, -156.8)
    (2.449, -160.4)
    (2.531, -163.6)
    (2.612, -166.3)
    (2.694, -168.6)
    (2.776, -170.5)
    (2.857, -172.1)
    (2.939, -173.4)
    (3.020, -174.5)
    (3.102, -175.5)
    (3.184, -176.3)
    (3.265, -176.9)
    (3.347, -177.4)
    (3.429, -177.9)
    (3.510, -178.2)
    (3.592, -178.5)
    (3.673, -178.8)
    (3.755, -179.0)
    (3.837, -179.2)
    (3.918, -179.3)
    (4.000, -179.4)
  };


\end{tikzpicture}

\end{center}
\end{minipage}



\end{document}