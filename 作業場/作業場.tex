\documentclass[a4paper,12pt]{article}
%\documentclass[fleqn]{article}

% ---パッケージ---
\usepackage{amsmath,amssymb}    %数式用
\usepackage{tcolorbox}   %囲み枠用(tcolorboxに変更)
\usepackage{geometry}   %余白調節
\usepackage{tikz}  % ← 図を描くためのTikZパッケージ
\geometry{margin=25mm}  %余白を少し狭く
\usetikzlibrary{decorations.pathmorphing,patterns,positioning,arrows.meta} % バネ・壁の模様
\tikzset{
  block/.style = {draw, rectangle, minimum height=2em, minimum width=3em},
  sum/.style = {draw, circle, inner sep=0pt, minimum size=5mm},
  input/.style = {coordinate},
  output/.style = {coordinate}
}
\usetikzlibrary{calc}


% --- 日本語用パッケージ ---
\usepackage{luatexja}         % 日本語表示に必要
\usepackage{luatexja-fontspec} % フォント指定用

% --- フォント指定(Overleaf標準フォント)---
\setmainjfont{IPAexMincho}  % 明朝体
%\setmainjfont{IPAexGothic}  % ゴシック体にしたい場合

% --- tcolorbox の設定 ---
\tcbset{
    colframe=black,
    colback=white,         % 本文の背景(白)
    boxrule=0.8pt,
    arc=3pt,
    outer arc=3pt,
    boxsep=4pt,
    coltitle=black,
    colbacktitle=gray!20,  % タイトルの背景(グレー)
    fonttitle=\normalsize
}

\begin{document}

% ---------------[5(5)]--------------- 済
\begin{tcolorbox}[title={5. (5)\(m=1,d_1=1,d_2=1,k=1\)とし、入力変位\(x_1(t)=\sin(t)\)を与えたときの応
\indent \quad 答を求めよ。 }]

パラメータを代入し、\(x_1(t)=\sin(t)\)のラプラス変換は\(X_i(s)=\dfrac{1}{s^2+1}\)なので
    \vspace{-4mm}
    \begin{align*}
        &\qquad G(s) = \frac{s}{s^2 + 2s + 1} \\
        &\therefore \quad X(s) = G(s) F(s) = \frac{s}{(s+1)^2(s^2 +1)} \\
        &\therefore \quad X(s) =  \frac{\left(-\frac{1}{2}\right)}{(s+1)^2}
        + \frac{\left(\frac{1}{2}\right)}{s^2 +1}\\
        &\therefore \quad \mathcal{L}^{-1} \left[ X(s)\right] 
        = \mathcal{L}^{-1} \left[\frac{\left(-\frac{1}{2}\right)}{(s+1)^2}\right]
        + \mathcal{L}^{-1} \left[\frac{\left(\frac{1}{2}\right)}{s^2 +1}\right] \\
        &\therefore \quad x(t) = -\frac{1}{2} t e^{-t} + \frac{1}{2} \sin t
    \end{align*}

\end{tcolorbox}

% ---------------[5(6)]--------------- 済
\begin{tcolorbox}[title={5. (6)\(m=1,d_1=1,d_2=2,k=2\)とし、入力変位\(x_1(t)=\sin(2t)\)を与えたときの応
\indent \quad 答を求めよ。 }]

パラメータを代入し、\(x_1(t)=\sin(2t)\)のラプラス変換は\(X_i(s)=\dfrac{2}{s^2+2^2}\)なので
    \vspace{-4mm}
    \begin{align*}
        &\qquad G(s) = \frac{s}{s^2 + 3s + 2} \\
        &\therefore \quad X(s) = G(s) F(s) = \frac{s}{(s+1)(s+2)(s^2+2^2)} \\
        &\therefore \quad X(s) =  \frac{\left(-\frac{1}{5}\right)}{s+1} 
        + \frac{\left(\frac{1}{4}\right)}{s+2}
        + \frac{\left(-\frac{1}{20}s+\frac{3}{10}\right)}{s^2+2^2}\\
        &\therefore \quad \mathcal{L}^{-1} \left[ X(s)\right] 
        = \mathcal{L}^{-1} \left[\frac{\left(-\frac{1}{5}\right)}{s+1}  \right]
        + \mathcal{L}^{-1} \left[\frac{\left(\frac{1}{4}\right)}{s+2} \right]
        + \mathcal{L}^{-1} \left[\frac{\left(-\frac{1}{20}s+\frac{3}{10}\right)}{s^2+2^2}\right]  \\
        &\therefore \quad x(t) = -\frac{1}{5} e^{-t} + \frac{1}{4} e^{-2t} - \frac{1}{20} \left\{ \cos (2t) - 3\sin (2t)\right\}
    \end{align*}

\end{tcolorbox}

% ---------------[5(7)]--------------- 済
\begin{tcolorbox}[title={5. (7)\(m=1,d_1=2,d_2=0,k=2\)とし、ステップ応答を求めよ。 }]

    パラメータを代入し、ステップ入力のラプラス変換は\(F(s)=\dfrac{1}{s}\)なので
    \vspace{-4mm}
    \begin{align*}
        &\qquad G(s) = \frac{2s}{s^2 + 2s + 2} \\
        &\therefore \quad X(s) = G(s) F(s) 
        = \frac{2}{s^2 + 2s + 2} \\
        &\therefore \quad \mathcal{L}^{-1} \left[ X(s)\right] 
        = \mathcal{L}^{-1} \left[\frac{2}{(s+1)^2+1}\right] \\
        &\therefore \quad x(t) = 2 e^{-t} \sin t
    \end{align*}

\end{tcolorbox}



\end{document}